~\vfill

\begin{minipage}{1.0\textwidth}

\thispagestyle{empty}
\setlength{\parindent}{0pt}
\setlength{\parskip}{\baselineskip}

\noindent\rule{1.0\textwidth}{0.4pt}

Deze handleiding is zorgvuldig geproduceerd. Desalniettemin kunnen de auteur en uitgever niet garanderen dat de informatie daarin vrij is van fouten. Lezers worden geadviseerd in gedachten te houden dat verklaringen, gegevens, illustraties, procedurele details of andere items per ongeluk onnauwkeurig kunnen zijn.

Tenzij anders vermeld is alles in dit werk gelicenceerd onder een Creative Commons Naamsvermelding-Gelijkdelen 4.0-licentie. Wanneer u gebruik wilt maken van dit werk, hanteer dan de volgende methode van naamsvermelding:

``I.A.W. Filot, P2000T SD-kaart handleiding, CC-BY-SA 4.0 gelicenseerd.'' 

\includegraphics[]{img/by-sa.png}

De volledige licentie-tekst is te lezen op:\\
\url{https://creativecommons.org/licenses/by-sa/4.0/}

De bronbestanden voor het produceren van deze handleiding zijn te vinden in onderstaande repository:\\
\faGithub\;\url{https://github.com/ifilot/P2000T-SD-kaart-handleiding}

Aanvullingen of opmerkingen met betrekking tot deze handleiding mogen ingediend worden als ``Issues'' in bovenstaande repository.

\noindent\rule{1.0\textwidth}{0.4pt}
Versie: 1.0.1\\
Compilatiedatum: \today

\end{minipage}